% ------------------------------------------------------------------------
% ------------------------------------------------------------------------
% Modelo UNIAVAN para Trabalhos Acadêmicos baseado em Modelo UFSC de 
% Alisson Lopes Furlani utilizando a classe abntex2
%
% Autor: Luiz Fernando Marquez Arruda
%
% Olá caro autor, leia o arquivo Leia-me.txt
%
%
% Atenciosamente 
%
% Luiz Fernando M. Arruda -  13/06/2024

% ------------------------------------------------------------------------
% ------------------------------------------------------------------------

\documentclass[% -- opções da classe uniavan --
	12pt,				% tamanho da fonte
	openright,			% capítulos começam em pág ímpar (insere página vazia caso preciso)
	oneside,			% para impressão no anverso. Oposto a twoside
	% -- opções da classe abntex2 --
	chapter=TITLE,		% títulos de capítulos convertidos em letras maiúsculas
	section=TITLE,		% títulos de seções convertidos em letras maiúsculas
	%subsection=TITLE,	% títulos de subseções convertidos em letras maiúsculas
	%subsubsection=TITLE,% títulos de subsubseções convertidos em letras maiúsculas
	% -- opções do pacote babel --
	english,			% idioma adicional para hifenização
	%french,			% idioma adicional para hifenização
	%spanish,			% idioma adicional para hifenização
	brazil				% o último idioma é o principal do documento
	]{setup/uniavan}

\addbibresource{references.bib} % Seus arquivos de referências

% -------------------------------------------------------------------------
% Inclusão de novas bibliotecas
% -------------------------------------------------------------------------
% Inclusão de novas bibliotecas devem ser feitas arquivo setup/my_packages.tex
% -----------------------------

\input{setup/my_packages}

% -------------------------------------------------------------------------
% Dados gerais do documento
% -------------------------------------------------------------------------
% Inclusão do nome do autor, título, orientador, membros da banca, curso etc
% Devem ser editados no arquivo setup/information.tex
% -----------------------------

% ---
% Informações de dados para CAPA e FOLHA DE ROSTO
% ---
% FIXME Substituir 'Nome completo do autor' pelo seu nome.
\autor{Nome completo do autor}
% FIXME Substituir 'Título do trabalho' pelo título da trabalho.
\titulo{Título do trabalho}
% FIXME Substituir 'Subtítulo (se houver)' pelo subtítulo da trabalho.  
% Caso não tenha substítulo, comente a linha a seguir.
\subtitulo{subtítulo (se houver)}
% FIXME Substituir 'XXXXXX' pelo nome do seu
% orientador.
\orientador{Prof. XXXXXX, Dr.}
% FIXME Se for orientado por uma mulher, comente a linha acima e descomente a linha a seguir.
% \orientador[Orientadora]{Nome da orientadora, Dra.}
% FIXME Substituir 'XXXXXX' pelo nome do seu
% coorientador. Caso não tenha coorientador, comente a linha a seguir.
%\coorientador{Prof. XXXXXX, Dr.}
% FIXME Se for coorientado por uma mulher, comente a linha acima e descomente a linha a seguir.
% \coorientador[Coorientadora]{XXXXXX, Dra.}
% FIXME Substituir '[ano]' pelo ano (ano) em que seu trabalho foi defendido.
\ano{[ano]}
% FIXME Substituir '[dia] de [mês] de [ano]' pela data em que ocorreu sua defesa.
\data{[dia] de [mês] de [ano]}
% FIXME Substituir 'Local' pela cidade em que ocorreu sua apresentação.
\local{[Local]}
% FIXME Substituir 'XXXXXX' pelo nome do coordenador do curso.
\coordenador{Prof. XXXXX, Dr.}
% FIXME Se for coordenador por mulher, comente a linha acima e descomente a linha a seguir.
%\coordenador[Coordenadora]{XXXXXX, Dra.}
% FIXME Substituir '[Sigla]' pela sigla da faculdade.
\instituicaosigla{UNIAVAN}
% FIXME Substituir '[Nome da Faculdade]' pelo nome da faculdade.
\instituicao{Centro Universitário Avantis}
% FIXME Substituir '[Tipo de Trabalho]' pelo tipo de trabalho (Trabalho de Conclusão de Curso, Tese, Dissertação). 
\tipotrabalho{[Tipo de Trabalho]}
% FIXME Substituir '[Título Escolar]' pela grau adequado (Doutor/Mestre/Especialista/Bacharel/Licenciatura/Tecnólogo em XX).
\formacao{[Formação] em [XX]}
% FIXME Substituir '[Nível]' pelo nível adequado (Tecnólogo / Graduação / Mestrado / Doutorado).
\nivel{[Nível]}
% FIXME Substituir '[Nome do curso]' pelo nome curso adequado.
\curso{[Nome do curso]}

\preambulo
{%
%\imprimirtipotrabalho~apresentado ao curso de~\imprimircurso~do~\imprimirinstituicao~de Balneário Camboriú para a obtenção do título de~\imprimirformacao~em~\imprimircurso.
\imprimirtipotrabalho~apresentado ao curso de~\imprimircurso~do~\imprimirinstituicao~de Balneário Camboriú para a obtenção do título de~\imprimirformacao.
}
% ---

% ---
% Configurações de aparência do PDF final
% ---
% alterando o aspecto da cor azul
\definecolor{blue}{RGB}{41,5,195}
% informações do PDF
\makeatletter
\hypersetup{
     	%pagebackref=true,
		pdftitle={\@title}, 
		pdfauthor={\@author},
    	pdfsubject={\imprimirpreambulo},
	    pdfcreator={LaTeX with abnTeX2},
		pdfkeywords={ufsc, latex, abntex2}, 
		colorlinks=true,       		% false: boxed links; true: colored links
    	linkcolor=black,%blue,          	% color of internal links
    	citecolor=black,%blue,        		% color of links to bibliography
    	filecolor=black,%magenta,      		% color of file links
		urlcolor=black,%blue,
		bookmarksdepth=4
}
\makeatother
% ---

% ---
% carregar e compilar a lista de abreviaturas e siglas e a lista de símbolos
% ---

% carrega a lista de abreviaturas e siglas e a lista de símbolos
% Declaração das siglas
\siglalista{ABNT}{Associação Brasileira de Normas Técnicas}
%\newacronym[user1=\emph{english}]{pt}{pt}{portugues}
%\newacronym[\glslongpluralkey={siglas}]{s}{s}{sigla}

% Declaração dos símbolos
\simbololista{C}{\ensuremath{C}}{Circunferência de um círculo}
\simbololista{pi}{\ensuremath{\pi}}{Número pi} 
\simbololista{r}{\ensuremath{r}}{Raio de um círculo}
\simbololista{A}{\ensuremath{A}}{Área de um círculo}

% Declaração de acrônimos
\glossariolista{Palavra}{Descrição da palavra... escrevendo aqui para ocupar mais de uma linha e testar o template}
\glossariolista{OutraPalavra}{Descrição da outra palavra}

% compila a lista de abreviaturas e siglas e a lista de símbolos
\makenoidxglossaries 

% ---

% ---
% compila o indice
% ---
\makeindex
% ---

% ----
% Início do documento
% ----
\begin{document}

% Seleciona o idioma do documento (conforme pacotes do babel)
%\selectlanguage{english}
\selectlanguage{brazil}

% Retira espaço extra obsoleto entre as frases.
\frenchspacing 

% Espaçamento 1.5 entre linhas
\OnehalfSpacing

% Corrige justificação
%\sloppy

% -------------------------------------------------------------------------
% ELEMENTOS PRÉ-TEXTUAIS
% -------------------------------------------------------------------------
% Capa, folha de rosto, ficha bibliográfica, errata, folha de aprovação
% Dedicatória, agradecimentos, epígrafe, resumos, listas
% -------------------------------------------------------------------------
% Capa
% -------------------------------------------------------------------------
\imprimircapa
% -------------------------------------------------------------------------


% -------------------------------------------------------------------------
% Folha de rosto
% (o * indica que haverá a ficha bibliográfica)
% -------------------------------------------------------------------------
\imprimirfolhaderosto
% -------------------------------------------------------------------------

% -------------------------------------------------------------------------
% Folha de aprovação
% -------------------------------------------------------------------------
% A folha de aprovação é feita em arquivo separado (para facilitar a impressão / eventual mudança de membro pelo suplente / inclusão da folha escaneada).
% O arquivo de edição da folha de aprovação  está em beforetext/aprovacao.tex
% -------------------------------------------------------------------------
%\begin{folhadeaprovacao}
	\OnehalfSpacing
	\centering
	\imprimirautor\\%
	\vspace*{10pt}		
	\textbf{\imprimirtitulo}%
	\ifnotempty{\imprimirsubtitulo}{:~\imprimirsubtitulo}\\%
	%		\vspace*{31.5pt}%3\baselineskip
	\vspace*{\baselineskip}
	%\begin{minipage}{\textwidth}
	O presente trabalho em nível de~\imprimirnivel~foi avaliado e aprovado por banca examinadora composta pelos seguintes membros:\\
	%\end{minipage}%
	\vspace*{\baselineskip}
	Prof.(a) xxxx, Dr(a).\\
	Instituição xxxx\\
	\vspace*{\baselineskip}
	Prof.(a) xxxx, Dr(a).\\
	Instituição xxxx\\
	\vspace*{\baselineskip}
	Prof.(a) xxxx, Dr(a).\\
	Instituição xxxx\\
	\vspace*{2\baselineskip}
	\begin{minipage}{\textwidth}
		Certificamos~que~esta~é~a~\textbf{versão~original~e~final}~do~\imprimirtipotrabalho~que~foi~julgado~adequado~para~obtenção~do~título~de~\imprimirformacao.\\
	\end{minipage}
	%    \vspace{-0.7cm}
	\vspace*{\fill}
	\assinatura{\OnehalfSpacing~\imprimircoordenador \\ \imprimircoordenadorRotulo~do~Curso de~\imprimirnivel~em~\imprimircurso}
	\vspace*{\fill}
	\assinatura{\OnehalfSpacing\imprimirorientador \\ \imprimirorientadorRotulo}
	%	\ifnotempty{\imprimircoorientador}{
	%	\assinatura{\imprimircoorientador \\ \imprimircoorientadorRotulo \\
	%		\imprimirinstituicao~--~\imprimirinstituicaosigla}
	%	}
	% \newpage
	\vspace*{\fill}
	\imprimirlocal, \imprimirano.
\end{folhadeaprovacao}
%\includepdf{beforetext/aprovacao.pdf}
% -------------------------------------------------------------------------


% -------------------------------------------------------------------------
% Elementos pré-textuais não obrigatórios
% -------------------------------------------------------------------------
% Os elementos pré-textuais devem ser editados na pasta beforetext
% -------------------------------------------------------------------------
% Agradecimentos
% -------------------------------------------------------------------------
%\begin{agradecimentos}

Elemento opcional. Deve aparecer o título centralizado, corpo 12, espacejamento 1,5, negrito e em letras maiúsculas. O texto deve ser justificado com espacejamento de 1,5, fonte 12 e dois espaços 1,5 entre o título.

\end{agradecimentos}
% -------------------------------------------------------------------------


% -------------------------------------------------------------------------
% Epígrafe
% -------------------------------------------------------------------------
%\begin{epigrafe}
	\vspace*{\fill}
    \begin{quoting}[rightmargin=0cm,leftmargin=8cm]
        \noindent A epígrafe é opcional. Apesar de ser escrita por outra pessoa, não deve vir entre aspas. Deve estar localizada na parte inferior direita da folha, utilizando-se fonte 10, espacejamento simples, alinhado a partir do meio da mancha (8 cm) para a margem direita. Referenciar o autor e ano sem parênteses.
		\newline
		\newline
		Nome do autor
    \end{quoting}
\end{epigrafe}
% -------------------------------------------------------------------------

% -------------------------------------------------------------------------
% Resumo
% -------------------------------------------------------------------------
% resumo em português
\setlength{\absparsep}{18pt} % ajusta o espaçamento dos parágrafos do resumo
\begin{resumo}
	\SingleSpacing
	A palavra RESUMO deve ser em fonte tamanho 12, letras maiúsculas, negrita e centralizada. Dois espaços simples entre o título e o texto. O texto deve ter alinhamento justificado, com fonte tamanho 10. Espacejamento simples e parágrafo único, sem recuo. O termo Palavras-chave deve ser negritado e em tamanho 12. As Palavras-chave devem ser separadas por ponto, espacejamento simples e parágrafo único, sem recuo.
	
	\textbf{Palavras-chave}: Palavra-chave 1. Palavra-chave 2. Palavra-chave 3.
\end{resumo}
% -------------------------------------------------------------------------


% -------------------------------------------------------------------------
% Abstract
% -------------------------------------------------------------------------
\begin{resumo}[Abstract]
	\SingleSpacing
	\begin{otherlanguage*}{english}
		The word ABSTRACT should be in font size 12, capital letters, bold and centered. Two simple spaces between the title and the text. The text should have justified alignment, with font size 10. Simple spacing and single paragraph, no indentation. The term Keywords should be bolded and in size 12. Keywords should be separated by dot, single spacing and single paragraph with no indentation.
		
		\textbf{Keywords}: Keyword 1. Keyword 2. Keyword 3.
	\end{otherlanguage*}
\end{resumo}
% -------------------------------------------------------------------------

% -------------------------------------------------------------------------
% LISTAS (não obrigatórias)
% -------------------------------------------------------------------------

% -------------------------------------------------------------------------
% Lista de figuras
% -------------------------------------------------------------------------
\imprimrilistadefiguras
% -------------------------------------------------------------------------
	
% -------------------------------------------------------------------------
% Lista de tabelas
% -------------------------------------------------------------------------
\imprimirlistadetabelas
% -------------------------------------------------------------------------

% -------------------------------------------------------------------------
% Lista de abreviaturas e siglas (devem ser declarados no preambulo)
% -------------------------------------------------------------------------
\imprimirlistadesiglas
% -------------------------------------------------------------------------

% -------------------------------------------------------------------------
% Lista de simbolos (devem ser declarados no preambulo)
% -------------------------------------------------------------------------
\imprimirlistadesimbolos
% -------------------------------------------------------------------------

% -------------------------------------------------------------------------
% Sumário
% -------------------------------------------------------------------------
\imprimirsumario
% -------------------------------------------------------------------------	

% ----------------------------------------------------------
% ELEMENTOS TEXTUAIS
% ----------------------------------------------------------
\textual

% ---

% ---
% ----------------------------------------------------------
\chapter{Introdução} \label{cap:Introducao}
% ----------------------------------------------------------

A primeira parte dos elementos textuais é a contextualização do trabalho. Esta seção deve possuir referências bibliográficas (sempre buscando diferentes autores). É neste momento que você estará apresentando o seu trabalho e indicando o contexto em que ele se encontra.
Você pode iniciar colocando o leitor a par dos conceitos e tecnologias explorados ao longo do trabalho. Caso o seu trabalho não possua a seção de fundamentação teórica (Capítulo \ref{cap:Fundamentacao}), os conceitos e tecnologias devem ser melhor aprofundados nesta seção.
Em seguida você pode explicar qual é o problema que o projeto pretende resolver com a solução proposta no objetivo geral.
Ao final desta seção, você irá dizer algo como:
Dentro deste contexto, este trabalho procura fazer uma contribuição na área de .... através do desenvolvimento e avaliação de...

\section{Objetivos}\label{sec:objetivos}
% ----------------------------------------------------------

Esta seção formaliza os objetivos do trabalho, conforme descrito a seguir.

% ----------------------------------------------------------
\subsection{Objetivo Geral}\label{subsec:objetivosgerais}
% ----------------------------------------------------------

Procure utilizar apenas uma frase para descrever o objetivo geral, iniciando com um verbo no infinitivo. Evite muitos conectores e explicações, pois eles não fazem parte do objetivo geral e já constituem parte dos objetivos específicos.

% ----------------------------------------------------------
\subsection{Objetivos Específicos}
% ----------------------------------------------------------
\begin{itemize}
    \item Esta seção é uma lista de itens (como esta), cada um sendo um objetivo. É interessante que esses objetivos sejam numerados de alguma forma (o propósito desta numeração não é criar uma ordem de importância, mas permitir que o objetivo possa ser referenciado ao longo do projeto);
    
    \item Procure ser realista e não escreva objetivos muito gerais ou muito abertos;
    
    \item Evite listar muitos objetivos específicos e colocar como objetivos específicos “O estudo ou aprofundamento de alguma coisa”. O estudo é um meio para alcançar o seu objetivo;
    
    \item Você deve evitar o preenchimento de uma sequência de atividades que será realizada (ver metodologia). Essa sequência de atividades é o plano de trabalho e mostra como você irá trabalhar para alcançar os objetivos definidos aqui;

    \item Evite objetivos pessoais e procure focar em objetivos do trabalho;
    
    \item Cada um dos objetivos específicos deverá ser trabalhado mais tarde nas conclusões da dissertação, pois será preciso indicar como estes objetivos foram alcançados e, caso contrário, justificar o porquê do não atendimento a um objetivo traçado no início da pesquisa.

\end{itemize}

% ----------------------------------------------------------
\subsection{Justificativa}
% ----------------------------------------------------------

Aqui, o foco está em justificar a solução proposta. Você deve deixar muito claro para o leitor qual será a efetiva contribuição que o seu trabalho irá oferecer, procurando responder no seu texto a perguntas como, por exemplo:

Qual é a relevância da solução da proposta?

Qual é a complexidade da solução proposta?

Qual é a aplicabilidade da solução?

A solução é viável?

Qual é o seu diferencial a outros similares?

Qual é a motivação para ele?

Procure utilizar referências bibliográficas para ajudar na defesa da relevância da solução proposta.

A justificativa, como o próprio nome indica, é a argumentação a favor da validade da realização do trabalho proposto, identificando as contribuições esperadas.

% ----------------------------------------------------------
\subsection{LIMITAÇÃO DO ESCOPO}
% ----------------------------------------------------------

Nesta seção, você deve estabelecer os limites do trabalho, deixando claro para o leitor o escopo da pesquisa a ser realizada. Você deve identificar aquilo que será feito e aquilo que não será feito, ou seja, as limitações do trabalho. Procure ser o mais honesto possível. Evite criar expectativas que ultrapassem a capacidade do trabalho.

% ----------------------------------------------------------
\subsection{METODOLOGIA}
% ----------------------------------------------------------

Você deve iniciar esta seção classificando a sua pesquisa sob três pontos de vista: natureza (básica ou aplicada), objetivos (exploratória, descritiva ou explicativa) e forma de abordagem do problema (quantitativa ou qualitativa). Nas referências deste modelo, há duas bibliografias \cite{gil2008metodos, da2005metodologia} que podem ser utilizadas para você fundamentar a classificação.

Em seguida você deve descrever os caminhos que foram percorridos (procedimentos metodológicos) para se chegar aos objetivos propostos (levantamento, estudo de caso, pesquisa bibliográfica, dentre outros), por exemplo: 

Pesquisa bibliográfica: inicialmente, foi realizada uma pesquisa bibliográfica em diversas bases de dados para adquirir familiaridade com o tema Computacional (PC). A pesquisa explorou definições e características do PC e também buscou identificar as estratégias que estão sendo utilizadas para o seu desenvolvimento com diferentes tipos de públicos.

Desenvolvimento do jogo: esta etapa atendeu ao Objetivo Específico 1 do TCC. Como primeira atividade, foi realizada uma pesquisa bibliográfica sobre Teoria da Computação e Lógica de Computabilidade, através da qual foram definidos o enredo e a mecânica do jogo proposto. Em seguida, foram desenvolvidas atividades de documentação (Game Design Document), modelagem (diagramas UML), implementação e teste de usabilidade.

Criação do conjunto de problemas do jogo: esta etapa atendeu ao Objetivo Específico 2 do TCC. A primeira atividade realizada foi o estabelecimento de parâmetros para a avaliação da complexidade de modelos de autômatos finitos e de máquina de Turing. Em seguida foi criado um conjunto de 23 problemas que exploram o poder dos autômatos finitos e da máquina de Turing de maneira incremental, tendo como base os parâmetros estabelecidos. Na última atividade desta etapa, os 23 problemas foram implementados no banco de dados do jogo.

% ----------------------------------------------------------
\subsection{ESTRUTURA DO TRABALHO}
% ----------------------------------------------------------
Nesta seção você deve descrever a estrutura do TCC, falando um pouco sobre o conteúdo de cada capítulo, por exemplo:

Este trabalho está estruturado em cinco capítulos. O Capítulo \ref{cap:Introducao} é dividido em cinco seções e contextualiza o trabalho, traz também os objetivos a serem alcançados, a justificativa do projeto e a limitação do escopo, além da metodologia aplicada na sua elaboração.

O Capítulo \ref{cap:Fundamentacao} apresenta um estudo da literatura que explora os temas relacionados com o trabalho. A Seção \ref{sec:conceitos} trata...

O Capítulo \ref{cap:desenvolvimento} apresenta, em detalhes, o desenvolvimento da solução proposta na Seção \ref{subsec:objetivosgerais}. A Seção \ref{sec:visao} apresenta...

O Capítulo \ref{cap:teste} apresenta a avaliação...

O Capítulo \ref{cap:consideracoes} apresenta as considerações finais do trabalho, bem como as contribuições e trabalhos futuros.

% ----------------------------------------------------------
\chapter{FUNDAMENTAÇÃO TEÓRICA} \label{cap:Fundamentacao}
% ----------------------------------------------------------
Para o desenvolvimento deste capítulo, como o próprio título sugere, é importante o uso de referências bibliográficas!

A fundamentação teórica do trabalho tem a finalidade de descrever os conceitos e tecnologias utilizados no desenvolvimento (Capítulo \ref{cap:desenvolvimento}). A estrutura de seções deste capítulo varia em função das características de cada trabalho, e deve ser definida junto com o orientador nos primeiros encontros da disciplina.

Evite utilizar citações diretas, especialmente citações com recuo (mais de 3 linhas). O uso exagerado deste tipo de citação revela a falta das habilidades de síntese e escrita. As citações diretas devem ser utilizadas em casos absolutamente necessários, e devem conter, além do ano de publicação, a página que o texto foi extraído.

Você também deve evitar a citação de um único autor ao longo do texto, por exemplo:
Segundo \textcite{da2005metodologia}, ….

Os sistemas de informação…para cada caso \cite{da2005metodologia} .

\textcite{da2005metodologia} entendem que…

Isto pode configurar plágio, ainda que citado o autor!

O caderno “Metodologia de Pesquisa Científica”, disponível no material de apoio da disciplina, explica como fazer citações diretas e indiretas conforme as normas da \gls{ABNT}.

Cabe destacar que este capítulo não é obrigatório. No entanto, caso ele não esteja presente no TCC, os conceitos de tecnologias utilizados no desenvolvimento devem estar bem aprofundados na introdução (Capítulo \ref{cap:Introducao}).

% ----------------------------------------------------------
\section{CONCEITOS EXPLORADOS NO TRABALHO}\label{sec:conceitos}
% ----------------------------------------------------------

Para cada conceito explorado no trabalho, você deve criar nova uma seção como esta, por exemplo: “2.1 INTERNET DAS COISAS”.

% ----------------------------------------------------------
\section{TECNOLOGIAS UTILIZADAS NO DESENVOLVIMENTO}\label{sec:tecnologias}
% ----------------------------------------------------------

Para cada tecnologia utilizada no desenvolvimento, você deve criar uma nova seção como esta, por exemplo: “2.2 PLATAFORMA ARDUINO”.

% ----------------------------------------------------------
\section{Exemplo citação longa em Látex} \label{}
% ----------------------------------------------------------


\begin{citacao}
	Após a ilustração, na parte inferior, indicar a fonte consultada (elemento obrigatório, mesmo que seja produção do próprio autor), legenda, notas e outras informações necessárias à sua compreensão (se houver). A ilustração deve ser citada no texto e inserida o mais próximo possível do texto a que se refere. \cite[p. 11]{gil2008metodos}.
\end{citacao}

% ----------------------------------------------------------
\section{Exemplo Equações e fórmulas em Látex}
% ----------------------------------------------------------

As equações e fórmulas devem ser destacadas no texto para facilitar a leitura. Para numerá-las, usar algarismos arábicos entre parênteses e alinhados à direita. Pode-se adotar uma entrelinha maior do que a usada no texto.

Exemplos, \ref{eq:Eq_1} e \ref{eq:Eq_2}.

\begin{equation}\label{eq:Eq_1}
C = 2 \pi r
\end{equation}

\begin{equation}\label{eq:Eq_2}
\gls{A} = \gls{pi} \gls{r}^2
\end{equation}


% ----------------------------------------------------------
\section{Exemplo Código-Fonte}
% ----------------------------------------------------------

Os trechos de código devem ser exibidos em formatação de código com linhas enumeradas sequenciais a esquerda para falicitar os comentários. Abaixo segue exemplos carregando o código através de um arquivo e digitando diretamente no texto.

\lstinputlisting[
    language=C, % Defina a linguagem do código
    numbers=left, % Exibir números de linha à esquerda
    %linerange={23-66}, % Defina os intervalos de linhas
    firstnumber={1}, % Números iniciais correspondentes aos intervalos
    stepnumber=1, % Incremento do número de linha
    caption={Exemplo carregando arquivo...},
    label=src:sample_c
]{sources/sample.c}

\begin{lstlisting}[language=json, caption={Exemplo de Dados JSON}, label={src:json}]
    {
        "name": "John Doe",
        "age": 30,
        "address": {
            "street": "1234 Main St",
            "city": "Anytown",
            "state": "CA",
            "zip": "12345"
        },
        "phoneNumbers": [
            {"type": "home", "number": "555-1234"},
            {"type": "work", "number": "555-5678"}
        ]
    }
    \end{lstlisting}


\begin{lstlisting}[language=C, caption={Exemplo digitado no texto}, label={src:codigo_2}]
	#include <stdio.h>
	
	void main(){
		printf("Olá Mundo!");
	}
	\end{lstlisting}

Exemplos, \ref{src:sample_c}, \ref{src:json} e \ref{src:codigo_2}.


% ----------------------------------------------------------
\chapter{DESENVOLVIMENTO} \label{cap:desenvolvimento}
% ----------------------------------------------------------

A estrutura de seções deste capítulo varia em função das características de cada trabalho, e deve ser definida junto com o orientador no decorrer da disciplina. A seguir é apresentada uma estrutura de seções tradicionalmente utilizada em TCCs que envolvem o desenvolvimento de um software. 


% ----------------------------------------------------------
\section{VISÃO GERAL DO PROJETO}\label{sec:visao}
% ----------------------------------------------------------

Em alguns casos, pode ser interessante fornecer ao leitor uma visão geral do projeto, especialmente quando a solução é complexa e/ou envolve diversos componentes. Você também pode utilizar esta seção para falar um pouco sobre o modelo de processo de software adotado (cascata, espiral, incremental, …) e o planejamento das atividades realizadas.

\begin{figure}[htb]
	\begin{center}
		\includegraphics{images/figura1.png}
	\end{center}
	\caption{\label{fig:Fig_1}Visão geral do projeto}
	\fonte{Elaborado pelo autor deste trabalho (2018).}
\end{figure}

A Figura \ref{fig:Fig_1} representa ...

% ----------------------------------------------------------
\section{LEVANTAMENTO DE REQUISITOS} \label{sec:requisitos}
% ----------------------------------------------------------

Você pode iniciar esta seção explicando como e quando foram levantados os requisitos do sistema. Entrevistas com os proprietários da empresa? Documentação do software legado? Questionários aplicados aos usuários?

Em seguida você deve apresentar a especificação dos Requisitos Funcionais, Requisitos Não Funcionais e Regras de Negócio do sistema, conforme os ensinamentos da disciplina de Engenharia de Software, por exemplo:

A partir das entrevistas com os proprietários da empresa, foram identificados os seguintes requisitos funcionais para o sistema a ser desenvolvido:

\textbf{RF01} – O sistema deverá permitir ao usuário manter produtos;

\textbf{RF02} – O sistema deverá permitir ao usuário administrador manter categorias de produtos;

\textbf{RF03} – ...

Os seguintes requisitos não funcionais:

\textbf{RNF01} – Todas as funcionalidades serão executadas online, ou seja, através de acesso a um servidor web;

\textbf{RNF02} – Os dados serão armazenados em banco de dados MySQL;

\textbf{RNF03} – As linguagens para implementação são: HTML5, CSS, Javascript, jQuery e PHP;

\textbf{RNF04} – A interface gráfica com o usuário deve ser compatível com telas de computadores desktop, tablets e smartphones, e empregar o conceito de Web Design Responsivo através do framework Bootstrap.

\textbf{RNF05} – ...

E as seguintes regras de negócio:

\textbf{RN01} – A venda a prazo só poderá ser feita para clientes adimplentes;

\textbf{RN02} – ...

% ----------------------------------------------------------
\section{MODELAGEM}
% ----------------------------------------------------------

A estrutura de subseções a seguir varia em função das características de cada trabalho, e deve ser definida junto com o orientador no decorrer da disciplina. Os digramas comumente utilizados em TCCs que envolvem o desenvolvimento de um software são: diagrama de casos de uso, modelo de dados, diagrama de classes, diagrama de atividades, diagrama de sequência e diagrama de componentes. Normalmente, dois ou três desses diagramas são suficientes para fornecer as visões necessárias do projeto.

% ----------------------------------------------------------
\subsection{Casos de uso} 
% ----------------------------------------------------------

Caso o seu projeto utilize o diagrama de casos de uso (Figura \ref{fig:Fig_2}), é importante que ele esteja coerente com os requisitos funcionais (RFs) apresentados no levantamento de requisitos (Seção \ref{sec:requisitos}). Também é importante utilizar corretamente as notações UML, tais como “include”, “extend” e “generalization”. Não se esqueça de explicar o diagrama após a ilustração, conforme o exemplo a seguir.

\begin{figure}[htb]
	\begin{center}
		\includegraphics{images/figura2.png}
	\end{center}
	\caption{\label{fig:Fig_2}Diagrama de casos de uso}
	\fonte{Elaborado pelo autor deste trabalho com o uso da ferramenta StarUML (2018).}
\end{figure}

O usuário do tipo administrador herda as funcionalidades do usuário comum... 

Para executar as funcionalidades, os usuários devem realizar o login...

A documentação dos casos de uso encontra-se no Apêndice A deste trabalho.

% ----------------------------------------------------------
\subsection{Modelos de dados}
% ----------------------------------------------------------

O modelo de dados é um diagrama que descreve o esquema do banco de dados. Caso o seu projeto utilize este tipo de diagrama, não se esqueça de explicá-lo após a ilustração, conforme o exemplo a seguir.

\begin{figure}[htb]
    \begin{center}
	    \includegraphics{images/figura3.png}
	\end{center}
	\caption{\label{fig:Fig_3}Modelo de banco de dados}
	\fonte{Elaborado pelo autor deste trabalho com o uso da ferramenta StarUML (2018).}
\end{figure}

O esquema de banco de dados é composto de duas tabelas...Os campos do tipo “id” são utilizados para...

O código SQL de construção do esquema de banco de dados encontra-se no Apêndice B deste trabalho.

% ----------------------------------------------------------
\subsection{IMPLEMENTAÇÃO}
% ----------------------------------------------------------

Nesta seção você pode falar um pouco sobre o código desenvolvido. Não é necessário explicar ou apresentar todo o código fonte da aplicação. Você pode focar nas principais classes ou funções. É importante explicar quais foram as ferramentas utilizadas e o porquê da escolha de cada uma delas.

Você deve colocar o código na pasta sources e carrega-lo ao apêndice, usando aqui apenas uma referência ou trechos de código.

O código presente no Apêndice \ref{apendice:b} representa ....
% ----------------------------------------------------------
\chapter{TESTE OU AVALIAÇÃO} \label{cap:teste}
% ----------------------------------------------------------

Testes e avaliações tem o poder de enriquecer o trabalho acadêmico, fornecendo dados que permitirão ao leitor avaliar a qualidade da solução desenvolvida. Este capítulo pode apresentar, por exemplo, um teste de usabilidade com três seções: 4.1 PLANEJAMENTO, 4.2 EXECUÇÃO e 4.3 RESULTADOS.
Como outro exemplo, este capítulo pode apresentar um estudo de caso ou simulação com o uso da solução desenvolvida. Neste caso, uma possível estrutura de seções seria: 4.1 AMBIENTE DE ESTUDO, 4.2 IMPLANTAÇÃO, 4.3 RESULTADOS.
Para apresentação de dados ou estatísticas, utilize tabelas, lembrando que, diferente das ilustrações, as legendas das tabelas aparecem na parte superior.

\begin{table}[htb]
	\ABNTEXfontereduzida
	\caption{\label{tab:Tab_1}Exemplo de tabela em Látex.}
	\begin{tabular}{@{}p{3.0cm}p{1.5cm}p{2cm}p{2.5cm}p{2.5cm}p{2.5cm}@{}}
		\toprule
		\textbf{Média concentração urbana} & \multicolumn{2}{l}{\textbf{População}} & \textbf{Produto Interno Bruto – PIB (bilhões R\$)} & \textbf{Número de empresas} & \textbf{Número de unidades locais} \\ \midrule
		\textbf{Nome}                      & \textbf{Total}   & \textbf{No Brasil}  &                                                   &                             & \\
		Ji-Paraná (RO)                     & 116 610          & 116 610             & 1,686                                             & 2 734                       & 3 082 \\
		Parintins (AM)                     & 102 033          & 102 033             & 0,675                                             & 634                         & 683 \\
		Boa Vista (RR)                     & 298 215          & 298 215             & 4,823                                             & 4 852                       & 5 187 \\
		Bragança (PA)                      & 113 227          & 113 227             & 0,452                                             & 654                         & 686 \\ \bottomrule
	\end{tabular}
	\fonte{Elaborado pelo autor deste trabalho (2018).}
\end{table}
% ----------------------------------------------------------
\chapter{CONSIDERAÇÕES FINAIS} \label{cap:consideracoes}
% ----------------------------------------------------------

Você deve iniciar as considerações “olhando” para os objetivos apresentados na Seção \ref{sec:objetivos}. Inicie falando sobre como os objetivos foram alcançados.
Em seguida fale sobre as suas experiências e descobertas ao realizar o trabalho, por exemplo, as vantagens e desvantagens das tecnologias utilizadas e as dificuldades encontradas no desenvolvimento da solução.
Encerre as considerações com narrativas mais gerais, expondo sua visão do trabalho como um todo.

% ----------------------------------------------------------
\section{CONTRIBUIÇÕES} 
% ----------------------------------------------------------

Quais foram as constituições do seu trabalho? É importante destacar que não são contribuições para você, mas sim para quem irá utilizar o trabalho como referência! Por exemplo, você pode citar como contribuições os estudos, especificações, modelos e outros recursos disponíveis no trabalho e que podem ser utilizados por terceiros como base para o desenvolvimento de novos trabalhos ou pesquisas.

% ----------------------------------------------------------
\section{TRABALHOS FUTUROS} 
% ----------------------------------------------------------

Listar o que pode ser melhorado ou adicionado na solução desenvolvida.

\gls{Palavra}

\gls{OutraPalavra}

% ----------------------------------------------------------
% ELEMENTOS PÓS-TEXTUAIS
% ----------------------------------------------------------
\postextual
% ----------------------------------------------------------

% ----------------------------------------------------------
% Referências bibliográficas
% ----------------------------------------------------------
\begingroup
    \printbibliography[title=REFERÊNCIAS]
\endgroup

% ----------------------------------------------------------
% Glossário
% ----------------------------------------------------------
\imprimirglossario

% ----------------------------------------------------------
% Apêndices
% ----------------------------------------------------------

% ---
% Inicia os apêndices
% ---
\begin{apendicesenv}
	%\partapendices* 
	\input{aftertext/apendice_a}
	% ----------------------------------------------------------
\chapter{Código Fonte}\label{apendice:b}
% ----------------------------------------------------------

%Carregando código fonte da pasta sources no apêndice 

\lstinputlisting[
    language=C, % Defina a linguagem do código
    numbers=left, % Exibir números de linha à esquerda
    firstnumber={1}, % Números iniciais correspondentes aos intervalos
    stepnumber=1, % Incremento do número de linha
    caption={Exemplo carregando arquivo...},
    label=src:sample_c
]{sources/sample.c}
\end{apendicesenv}
% ---


% ----------------------------------------------------------
% Anexos
% ----------------------------------------------------------

% ---
% Inicia os anexos
% ---
\begin{anexosenv}
	%\partanexos*
	% ----------------------------------------------------------
\chapter{Declaração de Isenção}\label{anexo:a}
% ----------------------------------------------------------

\begin{center}
    \textbf{DECLARAÇÃO DE ISENÇÃO}
\end{center}

\vspace{1cm}

\imprimirlocal,~\imprimirdata.

\vspace{1cm}

Declaro, para todos os fins de direito, que assumo total responsabilidade pelo aporte ideológico conferido ao presente trabalho, estando ciente do disposto, da Resolução CONSUN 46-2019 e, isentando o Centro Universitário Avantis, o Curso de Sistemas de Informação, a Banca Examinadora e o Orientador de Trabalho de Conclusão de Curso de toda e qualquer responsabilidade acerca do mesmo.\\
\vspace{5cm}
\begin{center}

\assinatura{\MakeTextUppercase{\textbf{\imprimirautor}}}
\end{center}


\end{anexosenv}

%---------------------------------------------------------------------
% INDICE REMISSIVO
%---------------------------------------------------------------------
% \phantompart
% \printindex
%---------------------------------------------------------------------

\end{document}
