% ----------------------------------------------------------
\chapter{Introdução} \label{cap:Introducao}
% ----------------------------------------------------------

A primeira parte dos elementos textuais é a contextualização do trabalho. Esta seção deve possuir referências bibliográficas (sempre buscando diferentes autores). É neste momento que você estará apresentando o seu trabalho e indicando o contexto em que ele se encontra.
Você pode iniciar colocando o leitor a par dos conceitos e tecnologias explorados ao longo do trabalho. Caso o seu trabalho não possua a seção de fundamentação teórica (Capítulo \ref{cap:Fundamentacao}), os conceitos e tecnologias devem ser melhor aprofundados nesta seção.
Em seguida você pode explicar qual é o problema que o projeto pretende resolver com a solução proposta no objetivo geral.
Ao final desta seção, você irá dizer algo como:
Dentro deste contexto, este trabalho procura fazer uma contribuição na área de .... através do desenvolvimento e avaliação de...

\section{Objetivos}\label{sec:objetivos}
% ----------------------------------------------------------

Esta seção formaliza os objetivos do trabalho, conforme descrito a seguir.

% ----------------------------------------------------------
\subsection{Objetivo Geral}\label{subsec:objetivosgerais}
% ----------------------------------------------------------

Procure utilizar apenas uma frase para descrever o objetivo geral, iniciando com um verbo no infinitivo. Evite muitos conectores e explicações, pois eles não fazem parte do objetivo geral e já constituem parte dos objetivos específicos.

% ----------------------------------------------------------
\subsection{Objetivos Específicos}
% ----------------------------------------------------------
\begin{itemize}
    \item Esta seção é uma lista de itens (como esta), cada um sendo um objetivo. É interessante que esses objetivos sejam numerados de alguma forma (o propósito desta numeração não é criar uma ordem de importância, mas permitir que o objetivo possa ser referenciado ao longo do projeto);
    
    \item Procure ser realista e não escreva objetivos muito gerais ou muito abertos;
    
    \item Evite listar muitos objetivos específicos e colocar como objetivos específicos “O estudo ou aprofundamento de alguma coisa”. O estudo é um meio para alcançar o seu objetivo;
    
    \item Você deve evitar o preenchimento de uma sequência de atividades que será realizada (ver metodologia). Essa sequência de atividades é o plano de trabalho e mostra como você irá trabalhar para alcançar os objetivos definidos aqui;

    \item Evite objetivos pessoais e procure focar em objetivos do trabalho;
    
    \item Cada um dos objetivos específicos deverá ser trabalhado mais tarde nas conclusões da dissertação, pois será preciso indicar como estes objetivos foram alcançados e, caso contrário, justificar o porquê do não atendimento a um objetivo traçado no início da pesquisa.

\end{itemize}

% ----------------------------------------------------------
\subsection{Justificativa}
% ----------------------------------------------------------

Aqui, o foco está em justificar a solução proposta. Você deve deixar muito claro para o leitor qual será a efetiva contribuição que o seu trabalho irá oferecer, procurando responder no seu texto a perguntas como, por exemplo:

Qual é a relevância da solução da proposta?

Qual é a complexidade da solução proposta?

Qual é a aplicabilidade da solução?

A solução é viável?

Qual é o seu diferencial a outros similares?

Qual é a motivação para ele?

Procure utilizar referências bibliográficas para ajudar na defesa da relevância da solução proposta.

A justificativa, como o próprio nome indica, é a argumentação a favor da validade da realização do trabalho proposto, identificando as contribuições esperadas.

% ----------------------------------------------------------
\subsection{LIMITAÇÃO DO ESCOPO}
% ----------------------------------------------------------

Nesta seção, você deve estabelecer os limites do trabalho, deixando claro para o leitor o escopo da pesquisa a ser realizada. Você deve identificar aquilo que será feito e aquilo que não será feito, ou seja, as limitações do trabalho. Procure ser o mais honesto possível. Evite criar expectativas que ultrapassem a capacidade do trabalho.

% ----------------------------------------------------------
\subsection{METODOLOGIA}
% ----------------------------------------------------------

Você deve iniciar esta seção classificando a sua pesquisa sob três pontos de vista: natureza (básica ou aplicada), objetivos (exploratória, descritiva ou explicativa) e forma de abordagem do problema (quantitativa ou qualitativa). Nas referências deste modelo, há duas bibliografias \cite{gil2008metodos, da2005metodologia} que podem ser utilizadas para você fundamentar a classificação.

Em seguida você deve descrever os caminhos que foram percorridos (procedimentos metodológicos) para se chegar aos objetivos propostos (levantamento, estudo de caso, pesquisa bibliográfica, dentre outros), por exemplo: 

Pesquisa bibliográfica: inicialmente, foi realizada uma pesquisa bibliográfica em diversas bases de dados para adquirir familiaridade com o tema Computacional (PC). A pesquisa explorou definições e características do PC e também buscou identificar as estratégias que estão sendo utilizadas para o seu desenvolvimento com diferentes tipos de públicos.

Desenvolvimento do jogo: esta etapa atendeu ao Objetivo Específico 1 do TCC. Como primeira atividade, foi realizada uma pesquisa bibliográfica sobre Teoria da Computação e Lógica de Computabilidade, através da qual foram definidos o enredo e a mecânica do jogo proposto. Em seguida, foram desenvolvidas atividades de documentação (Game Design Document), modelagem (diagramas UML), implementação e teste de usabilidade.

Criação do conjunto de problemas do jogo: esta etapa atendeu ao Objetivo Específico 2 do TCC. A primeira atividade realizada foi o estabelecimento de parâmetros para a avaliação da complexidade de modelos de autômatos finitos e de máquina de Turing. Em seguida foi criado um conjunto de 23 problemas que exploram o poder dos autômatos finitos e da máquina de Turing de maneira incremental, tendo como base os parâmetros estabelecidos. Na última atividade desta etapa, os 23 problemas foram implementados no banco de dados do jogo.

% ----------------------------------------------------------
\subsection{ESTRUTURA DO TRABALHO}
% ----------------------------------------------------------
Nesta seção você deve descrever a estrutura do TCC, falando um pouco sobre o conteúdo de cada capítulo, por exemplo:

Este trabalho está estruturado em cinco capítulos. O Capítulo \ref{cap:Introducao} é dividido em cinco seções e contextualiza o trabalho, traz também os objetivos a serem alcançados, a justificativa do projeto e a limitação do escopo, além da metodologia aplicada na sua elaboração.

O Capítulo \ref{cap:Fundamentacao} apresenta um estudo da literatura que explora os temas relacionados com o trabalho. A Seção \ref{sec:conceitos} trata...

O Capítulo \ref{cap:desenvolvimento} apresenta, em detalhes, o desenvolvimento da solução proposta na Seção \ref{subsec:objetivosgerais}. A Seção \ref{sec:visao} apresenta...

O Capítulo \ref{cap:teste} apresenta a avaliação...

O Capítulo \ref{cap:consideracoes} apresenta as considerações finais do trabalho, bem como as contribuições e trabalhos futuros.
