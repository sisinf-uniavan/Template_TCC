% ----------------------------------------------------------
\chapter{CONSIDERAÇÕES FINAIS} \label{cap:consideracoes}
% ----------------------------------------------------------

Você deve iniciar as considerações “olhando” para os objetivos apresentados na Seção \ref{sec:objetivos}. Inicie falando sobre como os objetivos foram alcançados.
Em seguida fale sobre as suas experiências e descobertas ao realizar o trabalho, por exemplo, as vantagens e desvantagens das tecnologias utilizadas e as dificuldades encontradas no desenvolvimento da solução.
Encerre as considerações com narrativas mais gerais, expondo sua visão do trabalho como um todo.

% ----------------------------------------------------------
\section{CONTRIBUIÇÕES} 
% ----------------------------------------------------------

Quais foram as constituições do seu trabalho? É importante destacar que não são contribuições para você, mas sim para quem irá utilizar o trabalho como referência! Por exemplo, você pode citar como contribuições os estudos, especificações, modelos e outros recursos disponíveis no trabalho e que podem ser utilizados por terceiros como base para o desenvolvimento de novos trabalhos ou pesquisas.

% ----------------------------------------------------------
\section{TRABALHOS FUTUROS} 
% ----------------------------------------------------------

Listar o que pode ser melhorado ou adicionado na solução desenvolvida.

\gls{Palavra}

\gls{OutraPalavra}