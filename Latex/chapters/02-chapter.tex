% ----------------------------------------------------------
\chapter{FUNDAMENTAÇÃO TEÓRICA} \label{cap:Fundamentacao}
% ----------------------------------------------------------
Para o desenvolvimento deste capítulo, como o próprio título sugere, é importante o uso de referências bibliográficas!

A fundamentação teórica do trabalho tem a finalidade de descrever os conceitos e tecnologias utilizados no desenvolvimento (Capítulo \ref{cap:desenvolvimento}). A estrutura de seções deste capítulo varia em função das características de cada trabalho, e deve ser definida junto com o orientador nos primeiros encontros da disciplina.

Evite utilizar citações diretas, especialmente citações com recuo (mais de 3 linhas). O uso exagerado deste tipo de citação revela a falta das habilidades de síntese e escrita. As citações diretas devem ser utilizadas em casos absolutamente necessários, e devem conter, além do ano de publicação, a página que o texto foi extraído.

Você também deve evitar a citação de um único autor ao longo do texto, por exemplo:
Segundo \textcite{da2005metodologia}, ….

Os sistemas de informação…para cada caso \cite{da2005metodologia} .

\textcite{da2005metodologia} entendem que…

Isto pode configurar plágio, ainda que citado o autor!

O caderno “Metodologia de Pesquisa Científica”, disponível no material de apoio da disciplina, explica como fazer citações diretas e indiretas conforme as normas da \gls{ABNT}.

Cabe destacar que este capítulo não é obrigatório. No entanto, caso ele não esteja presente no TCC, os conceitos de tecnologias utilizados no desenvolvimento devem estar bem aprofundados na introdução (Capítulo \ref{cap:Introducao}).

% ----------------------------------------------------------
\section{CONCEITOS EXPLORADOS NO TRABALHO}\label{sec:conceitos}
% ----------------------------------------------------------

Para cada conceito explorado no trabalho, você deve criar nova uma seção como esta, por exemplo: “2.1 INTERNET DAS COISAS”.

% ----------------------------------------------------------
\section{TECNOLOGIAS UTILIZADAS NO DESENVOLVIMENTO}\label{sec:tecnologias}
% ----------------------------------------------------------

Para cada tecnologia utilizada no desenvolvimento, você deve criar uma nova seção como esta, por exemplo: “2.2 PLATAFORMA ARDUINO”.

% ----------------------------------------------------------
\section{Exemplo citação longa em Látex} \label{}
% ----------------------------------------------------------


\begin{citacao}
	Após a ilustração, na parte inferior, indicar a fonte consultada (elemento obrigatório, mesmo que seja produção do próprio autor), legenda, notas e outras informações necessárias à sua compreensão (se houver). A ilustração deve ser citada no texto e inserida o mais próximo possível do texto a que se refere. \cite[p. 11]{gil2008metodos}.
\end{citacao}

% ----------------------------------------------------------
\section{Exemplo Equações e fórmulas em Látex}
% ----------------------------------------------------------

As equações e fórmulas devem ser destacadas no texto para facilitar a leitura. Para numerá-las, usar algarismos arábicos entre parênteses e alinhados à direita. Pode-se adotar uma entrelinha maior do que a usada no texto.

Exemplos, \ref{eq:Eq_1} e \ref{eq:Eq_2}.

\begin{equation}\label{eq:Eq_1}
C = 2 \pi r
\end{equation}

\begin{equation}\label{eq:Eq_2}
\gls{A} = \gls{pi} \gls{r}^2
\end{equation}


% ----------------------------------------------------------
\section{Exemplo Código-Fonte}
% ----------------------------------------------------------

Os trechos de código devem ser exibidos em formatação de código com linhas enumeradas sequenciais a esquerda para falicitar os comentários. Abaixo segue exemplos carregando o código através de um arquivo e digitando diretamente no texto.

\lstinputlisting[
    language=C, % Defina a linguagem do código
    numbers=left, % Exibir números de linha à esquerda
    %linerange={23-66}, % Defina os intervalos de linhas
    firstnumber={1}, % Números iniciais correspondentes aos intervalos
    stepnumber=1, % Incremento do número de linha
    caption={Exemplo carregando arquivo...},
    label=src:sample_c
]{sources/sample.c}

\begin{lstlisting}[language=json, caption={Exemplo de Dados JSON}, label={src:json}]
    {
        "name": "John Doe",
        "age": 30,
        "address": {
            "street": "1234 Main St",
            "city": "Anytown",
            "state": "CA",
            "zip": "12345"
        },
        "phoneNumbers": [
            {"type": "home", "number": "555-1234"},
            {"type": "work", "number": "555-5678"}
        ]
    }
    \end{lstlisting}


\begin{lstlisting}[language=C, caption={Exemplo digitado no texto}, label={src:codigo_2}]
	#include <stdio.h>
	
	void main(){
		printf("Olá Mundo!");
	}
	\end{lstlisting}

Exemplos, \ref{src:sample_c}, \ref{src:json} e \ref{src:codigo_2}.

